The COVID-19 pandemic has highlighted the need for rapid and accurate computational modelling that provides forecasts to public health authorities on how day-to-day interventions can control the spread of a novel virus. Although many different models have been proposed, score-driven and state-space models represent a class of models that has been particularly fruitful in matching the dynamics of infectious diseases due to their inherent flexibility and robustness. An application of these models is presented in the work \textit{COVID-19 Active Case Forecasts in Latin American Countries Using Score-Driven Models} by Contreras-Espinoza et al. (2023) where score-driven models were used to forecast active COVID-19 cases in multiple Latin American countries.\\

The first class of time series models for new COVID-19 cases forecasting employed by (Contreras-Espinoza et al., 2023) comprises score-driven models. These are observation-driven state-space models in which the dynamic parameters are observable quantities that are updated every time new information becomes available. In particular, (Contreras-Espinoza et al., 2023) used a score-driven model for the negative binomial distribution, incorporating local-level and trend components. They also further extended it by adding a weekly seasonal component for new COVID-19 cases. \\

The second set of models used in the paper are state-space models with unobserved components. (Contreras-Espinoza et al., 2023) extended the state-space model by adding a weekly seasonal component, and assuming a negative binomial distribution for the data-generating process which is robust against possible small numbers of newly reported cases. By using the new state-space model with unobserved components for the negative binomial distribution, they separately modelled the trend, seasonality, and seasonal components of new COVID-19 cases, and they studied the forecasting accuracy for new COVID-19 cases using alternative forecasting windows (7, 14 and 28 days). \\

The main purpose of this project is to discuss the work of Contreras-Espinoza et al., and to apply their research to a new context: the area served by the Bologna wastewater treatment plant (WWTP). This area covers the city of Bologna and its neighbouring municipalities, with a population of approximately 800,000. By applying the score-driven and state-space models to the COVID-19 case data from this region, I aim to assess the performance and adaptability of these models in a different epidemiological and geographical setting. \\

The rest of this report is organized as follows: Section 2 presents the data, the statistical models and the methodology used to assess statistical performance and forecasting capability; Section 3 presents the results related to the Bologna COVID-19 data; Section 4 presents a discussion of (Contreras-Espinoza et al., 2023); Section 5 presents the conclusions.