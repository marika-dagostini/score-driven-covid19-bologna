The paper by (Contreras-Espinoza et al., 2023) identifies score-driven models with weekly seasonal effects for the negative binomial distribution (SD1 and SD2) as providing the most accurate forecasts for COVID-19 cases, outperforming score-driven models without seasonal components (SDWS) and nonlinear state-space models with unobserved components (SS). It introduces the novel application of weekly seasonality components and the use of the negative binomial distribution for COVID-19 data. Additionally, the study emphasizes the potential of these models for forecasting pandemic cases in other geographical and socio-cultural contexts beyond Latin America. This seems reasonable given that the use of the COVID-19 cases from the Bologna WWTP service area confirmed the robustness of their approach, even though the forecasting quality is significantly lower than the ones reported in the article for all the models. \\
\vspace{1cm}

Nevertheless, the study by (Contreras-Espinoza et al., 2023) also has several limitations. First, it lacks details on how the computations are performed, such as the software and libraries used, which limits the reproducibility of the analysis. Second, no estimate for the dispersion parameter of the negative binomial ($\nu$) is reported for the SS model. The authors might have assumed a fixed dispersion parameter based on previous literature or preliminary analyses, thereby simplifying the model. However, such a choice is not described in the paper, limiting again the reproducibility of the study. Third, the reliance on the negative binomial distribution may not capture all aspects of COVID-19 case distributions, suggesting a need for exploring alternative score-driven discrete probability distributions. Fourth, the authors suggest that these models can inform public health decisions, such as implementing quarantines and organizing vaccination processes, thus aiding authorities in managing current and future pandemics. However, the studied models turned out to be not suitable for long-term predictions, which are often more useful in pandemic analysis. Fifth, data limitations due to inconsistent and inaccurate COVID-19 reporting in many Latin American countries could undermine the reliability of the forecasts (Leiva et al., 2023). 
